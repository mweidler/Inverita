% This file is part of INVERITA.
%
% INVERITA Personal Backup Software
% Copyright (C) 2012-2013  Marc Weidler <marc.weidler@web.de>,
%                          Ulrichstr. 12/1, 71672 Marbach, Germany.
%                          All rights reserved.
%
% INVERITA is free software: you can redistribute it and/or modify
% it under the terms of the GNU General Public License as published by
% the Free Software Foundation, either version 3 of the License, or
% (at your option) any later version.
%
% INVERITA is distributed in the hope that it will be useful,
% but WITHOUT ANY WARRANTY; without even the implied warranty of
% MERCHANTABILITY or FITNESS FOR A PARTICULAR PURPOSE.  See the
% GNU General Public License for more details.
%
% You should have received a copy of the GNU General Public License
% along with this program.  If not, see <http://www.gnu.org/licenses/>.


\chapter{Introduction}

\begin{note}
This is a test
\end{note}

\section{Summary}
INVERITA Personal Backup tool for Unix/Linux (in the following called INVERITA)
is a software for data backup on various media.

\begin{note}
This is a test This is a test This is a test This is a test This is a test This is a test
This is a test This is a test This is a test This is a test This is a test This is a test
This is a test This is a test This is a test This is a test This is a test This is a test
This is a test This is a test This is a test This is a test This is a test This is a test
This is a test This is a test This is a test This is a test This is a test This is a test
This is a test This is a test This is a test This is a test This is a test This is a test
\end{note}

NormaL Test and is the test a test or is it not a test, i dont know, but i think this is a test
 Test and is the test a test or is it not a test, i dont know, but i think this is a test

\begin{warning}
This is a test This is a test This is a test This is a test This is a test This is a test
This is a test This is a test This is a test This is a test This is a test This is a test
This is a test This is a test This is a test This is a test This is a test This is a test
This is a test This is a test This is a test This is a test This is a test This is a test
This is a test This is a test This is a test This is a test This is a test This is a test
This is a test This is a test This is a test This is a test This is a test This is a test
\end{warning}

And this is a text

The main features are:
\begin{enumerate}
  \item Backup data in snapshots
  \item Nice and user-friendly user interface
  \item Hard-link technology for each backup snapshot to minimize needed media space
  \item Integrity validation of backups
\end{enumerate}


\section{Requirements}
INVERITA is developed and optimized for Unix/Linux operating systems.
It needs a file system on backup medium, that supports hard-links and
meta information (user, permissions, dates).
Currently, only the 'ext4' and 'nfs' file systems are supported.
It will not work with FAT/VFAT file systems!


\section{Installation, desktop integration and theming}
After package installation, there should appear a new menu entry "Inverita" in the
"Utility" section. Inverita uses automatically the current desktop theme, icons,
fonts, etc.


\section{Parameters}
INVERITA can be started with parameters, e.g.
\begin{console}
$ inverita <option>=<value> <option2>=<value2> ...
\end{console}
where \monospace{<option>} and \monospace{<value>} can be one of the following:
\begin{attributes}
 \item[-iconTheme=<ThemeName>]
    If for whatever reason the current desktop theme is not used, one can force the
    use of a specific theme by setting the theme name with this parameter, e.g.
    \monospace{-iconTheme=Faenza}.

  \item[-menusHaveIcons=true|false]
    If for whatever reason the menu icons are not shown, one can force this by using
    this parameter, e.g. \monospace{-menusHaveIcons=true}.

  \item[-translation=yes|no|<locale>]
    If for whatever reason the translation of the text displayed by Inverita is not
    working properly, one can control the behaviour by setting the translation
    parameter:
    \begin{attributes}[50pt]
       \item[yes]    use translation and use the system default language (default)
       \item[no]     disable translation. English language texts will be used.
       \item[<locale>] use translation and use this locate identifier, e.g. de\_DE
    \end{attributes}
    Example: \monospace{-translation=de\_DE}
\end{attributes}


\section{License}
Copyright (C) 2012-2013\\
Marc Weidler <marc.weidler@web.de>,\\
Ulrichstr. 12/1, 71672 Marbach, Germany.\\
All rights reserved.\\

INVERITA is free software: you can redistribute it and/or modify
it under the terms of the GNU General Public License as published by
the Free Software Foundation, either version 3 of the License, or
(at your option) any later version.

INVERITA is distributed in the hope that it will be useful,
but WITHOUT ANY WARRANTY; without even the implied warranty of
MERCHANTABILITY or FITNESS FOR A PARTICULAR PURPOSE.  See the
GNU General Public License for more details.

You should have received a copy of the GNU General Public License
along with this program.  If not, see <http://www.gnu.org/licenses/>.



\chapter{Configuration}

\section{Introduction}
INVERITA is a text-to-HTML conversion tool for web writers. Markdown
allows you to write using an easy-to-read, easy-to-write plain text
format, then convert it to structurally valid XHTML (or HTML).

Thus, "Markdown" is two things: a plain text markup syntax, and a
software tool, written in Perl, that converts the plain text markup 
to HTML.

Markdown works both as a Movable Type plug-in and as a standalone Perl
script -- which means it can also be used as a text filter in BBEdit
(or any other application that supporst filters written in Perl).

Full documentation of Markdown's syntax and configuration options is
available on the web: "http://daringfireball.net/projects/markdown/".
(Note: this readme file is formatted in Markdown.)

\includegraphic{images/MainScreen.png}{Main dialog}{Label}


\section{Installation and Requirements}

Markdown requires Perl 5.6.0 or later. Welcome to the 21st Century.
Markdown also requires the standard Perl library module `Digest::MD5`. 


\section{Typical usage scenarios}

\subsection{Backup on external USB drive}
\includegraphic{images/scenario-usbdrive.pdf}{Scenario USB drive}{ScenarioUSB}

\subsection{Backup on internet cloud drive}
\includegraphic{images/scenario-clouddrive.pdf}{Scenario Cloud drive}{ScenarioCloud}

\subsection{Backup on network attached storage (NAS)}
\includegraphic{images/scenario-nasdrive.pdf}{Scenario NAS drive}{ScenarioNAS}


\section{misc}

Backup
A backup is a configuration of what data to back up, storage location,
encryption, includes, excludes and many more.

Backup snapshot
A backup snapshot represents the copy of the data on the backup media
for a specific backup execution and contains the data themself, meta data
and a list of cryptographic representation of each contained file.

Backup snapshot validation
A backup snapshot on the backup media can be validated at every time
without accessing having access to the original data. A regererated SHA-1
checksum of each file is compared to the SHA-1 checksum generated from the
original file at time of backup.

Manual snapshot validation
This can even be done without INVERITA. The SHA-1 checksums are stored
in the file 'digests' which is compatible with the Linux/Unix command 'sha1sum'.
Change to the snapshot directory to be validated and start sha1sum

\begin{console}
$ cd /media/BackupUSBDrive/@243423-42323
$ sha1sum -c digests
\end{console}

NOTE: You can simply access the latest snapshot of a backup by using the
      symbolic link 'latest' that points to the latest snapshot directory.

      $ cd /media/BackupUSBDrive/latest
      $ sha1sum -c digests

NOTE: If your backup media is a PC or NAS, you are even able to automate a periodic
      backup snapshot validation with the help of a cron job. In case of a fail,
      you can send an eMail to all interested people.

      Make sure, that the package 'sendmail' is configured properly on you PC or NAS
      where the script should run on!


\section{Movable Type}

Markdown works with Movable Type version 2.6 or later (including 
MT 3.0 or later).

\begin{enumerate}
\item  Copy the "Markdown.pl" file into your Movable Type "plugins"
    directory. The "plugins" directory should be in the same directory
    as "mt.cgi"; if the "plugins" directory doesn't already exist, use
    your FTP program to create it. Your installation should look like
    this:

        (mt home)/plugins/Markdown.pl

\item Once installed, Markdown will appear as an option in Movable Type's
    Text Formatting pop-up menu. This is selectable on a per-post basis.
    Markdown translates your posts to HTML when you publish; the posts
    themselves are stored in your MT database in Markdown format.

\item If you also install SmartyPants 1.5 (or later), Markdown will offer
    a second text formatting option: "Markdown with SmartyPants". This
    option is the same as the regular "Markdown" formatter, except that
    automatically uses SmartyPants to create typographically correct
    curly quotes, em-dashes, and ellipses. See the SmartyPants web page
    for more information: <http://daringfireball.net/projects/smartypants/>

\item To make Markdown (or "Markdown with SmartyPants") your default
    text formatting option for new posts, go to Weblog Config ->
    Preferences.
\end{enumerate}

Note that by default, Markdown produces XHTML output. To configure
Markdown to produce HTML 4 output, see "Configuration", below.


