% This file is part of INVERITA.
%
% INVERITA Personal Backup Software
% Copyright (C) 2012-2013  Marc Weidler <marc.weidler@web.de>,
%                          Ulrichstr. 12/1, 71672 Marbach, Germany.
%                          All rights reserved.
%
% INVERITA is free software: you can redistribute it and/or modify
% it under the terms of the GNU General Public License as published by
% the Free Software Foundation, either version 3 of the License, or
% (at your option) any later version.
%
% INVERITA is distributed in the hope that it will be useful,
% but WITHOUT ANY WARRANTY; without even the implied warranty of
% MERCHANTABILITY or FITNESS FOR A PARTICULAR PURPOSE.  See the
% GNU General Public License for more details.
%
% You should have received a copy of the GNU General Public License
% along with this program.  If not, see <http://www.gnu.org/licenses/>.


\chapter{Introduction}

\section{Summary}
INVERITA Personal Backup tool for Unix/Linux (in the following called INVERITA)
is a software for data backup on various media.

The main features are:
\begin{enumerate}
  \item Backup data in snapshots
  \item Nice and user-friendly user interface
  \item Hard-link technology for each backup snapshot to minimize needed media space
  \item Integrity validation of backups
\end{enumerate}


\chapter{Configuration}

\includegraphic[0.8]{images/MainScreen.png}{Main dialog}{Label}
\includegraphic[0.8]{images/MainScreen.png}{Main dialog}{Label2}


\section{Installation and Requirements}

\section{Typical usage scenarios}

\subsection{Backup on external USB drive}
\includegraphic{images/scenario-usbdrive.pdf}{Scenario USB drive}{ScenarioUSB}

\subsection{Backup on internet cloud drive}
\includegraphic{images/scenario-clouddrive.pdf}{Scenario Cloud drive}{ScenarioCloud}

\subsection{Backup on network attached storage (NAS)}
\includegraphic{images/scenario-nasdrive.pdf}{Scenario NAS drive}{ScenarioNAS}

\includegraphic{images/scenario-clouddrive-detailed.pdf}{Scenario Cloud drive detailed}{ScenarioCloudD}

\section{misc}

\topic{Backup}
A backup is a configuration of what data to back up, storage location,
encryption, includes, excludes and many more.

\topic{Backup snapshot}
A backup snapshot represents the copy of the data on the backup media
for a specific backup execution and contains the data themself, meta data
and a list of cryptographic representation of each contained file.

\topic{Backup snapshot validation}
A backup snapshot on the backup media can be validated at every time
without accessing having access to the original data. A regererated SHA-1
checksum of each file is compared to the SHA-1 checksum generated from the
original file at time of backup.

\topic{Manual snapshot validation}
This can even be done without INVERITA. The SHA-1 checksums are stored
in the file 'digests' which is compatible with the Linux/Unix command 'sha1sum'.
Change to the snapshot directory to be validated and start sha1sum

\begin{lstlisting}[style=console]
§ cd /media/BackupUSBDrive/@243423-42323
§ sha1sum -c digests
\end{lstlisting}

\topic{Automated snapshot validation}
If your backup media is a PC or NAS, you are even able to automate a periodic
backup snapshot validation with the help of a cron job. In case of a fail,
you can send an eMail to all interested people.

Make sure, that the package 'sendmail' is configured properly on you PC or NAS
where the script should run on!

\begin{info}
You can simply access the latest snapshot of a backup by using the
symbolic link 'latest' that points to the latest snapshot directory.
\end{info}

\begin{lstlisting}[style=console]
# change to the latest backup snapshot directory
§ cd path_to_backup_location/latest   
# check the file's integrity with sha1sum
§ sha1sum -c digest
\end{lstlisting}

See appendix \ref{app:valscript} on page \pageref{app:fdl} for an example script for automated
snapshot validation.


\begin{info}
This is a test info
\end{info}

\begin{warning}
This is a test warning
\end{warning}
