% This file is part of INVERITA.
%
% INVERITA Personal Backup Software
% Copyright (C) 2012-2013  Marc Weidler <marc.weidler@web.de>,
%                          Ulrichstr. 12/1, 71672 Marbach, Germany.
%                          All rights reserved.
%
% INVERITA is free software: you can redistribute it and/or modify
% it under the terms of the GNU General Public License as published by
% the Free Software Foundation, either version 3 of the License, or
% (at your option) any later version.
%
% INVERITA is distributed in the hope that it will be useful,
% but WITHOUT ANY WARRANTY; without even the implied warranty of
% MERCHANTABILITY or FITNESS FOR A PARTICULAR PURPOSE.  See the
% GNU General Public License for more details.
%
% You should have received a copy of the GNU General Public License
% along with this program.  If not, see <http://www.gnu.org/licenses/>.


\chapter{Getting started}

\section{Purpose of INVERITA}
INVERITA Personal Backup tool for Unix/Linux (in the following called INVERITA)
is a software for data backup on various media. It's intention is not to be
used for a complete system backup -- use a partition imaging software instead.
INVERITA should be used for backup your home or data directory/partition
and keep several historical snapshots to go back to previous file versions.

\section{Requirements}
INVERITA is developed and optimized for Unix/Linux operating systems. It needs a file
system on backup medium, that supports hard-links and meta information (user, permissions, dates).

\begin{warning}
Currently, only the 'ext4' and 'nfs' file systems are supported for backup mediums.
It will not work with FAT/VFAT file systems!
Make sure, that your file-system for backup data on your NFS server is 'ext4'.
The data to be backed up can be reside on every files system that are supported
by your Unix/Linux system.
\end{warning}

\section{Installation, desktop integration and theming}
INVERITA can be installed by downloading the deb-package or, even better,
including the Inverita PPA to your system. See details on <http://www.launchpad.net/~mweidler/inverita>.

After package installation, there should appear a new menu entry \textbf{Inverita} in the
\textit{Utility} section. Inverita uses automatically the current desktop theme, icons,
fonts, etc.

\section{Parameters}
INVERITA can be started with parameters, e.g.

\begin{lstlisting}[style=console]
§ inverita <option>=<value> <option2>=<value2> ...
\end{lstlisting}

where \monospace{<option>} and \monospace{<value>} can be one of the following:
\begin{attributes}
 \item[-iconTheme=<ThemeName>]
    If for whatever reason the current desktop theme is not used, one can force the
    use of a specific theme by setting the theme name with this parameter, e.g.
    \monospace{-iconTheme=Faenza}.

  \item[-menusHaveIcons=true|false]
    If for whatever reason the menu icons are not shown, one can force this by using
    this parameter, e.g. \monospace{-menusHaveIcons=true}.

  \item[-translation=yes|no|<locale>]
    If for whatever reason the translation of the text displayed by Inverita is not
    working properly, one can control the behaviour by setting the translation
    parameter:
    \begin{attributes}[50pt]
       \item[yes]    use translation and use the system default language (default)
       \item[no]     disable translation. English language texts will be used.
       \item[<locale>] use translation and use this locate identifier, e.g. de\_DE
    \end{attributes}
    Example: \monospace{-translation=de\_DE}
\end{attributes}


\section{Terminology}

This section explains important terms which is needed to understand
the functionality of Inverita.

\subsection{Backup}
   Inverita can manage several Backups. Each Backup has
         a storage location where the configuration and the
         Backup snapshots are stored.
         Inverita stores the list of Backup in the user's
         configuration directory \monospace{\$HOME/.config/inverita.conf}.

\label{test1}

\subsection{Configuration}
         The Backup Configuration specifies the name of the Backup,
         if the data should be encrypted or not, included
         directories, exclude patterns and many more options.
         The configuration file is stored in the backup storage
         location and has the name 'inverita.conf'.

\label{test2}

\subsection{Backup snapshot}
         The Backup snapshot is a directory inside the
         backup storage and contains meta data (number of files, sizes, etc.),
         cryptographic digests of all backed up files and the backed up files itself.
         Identical files are linked in all backup snapshots and thus do not
         require additional storage space.
         The directory names of backup snapshots start with a @-sign followed by
         the time stamp when the backup snapshot has been created, e.g. \monospace{@2013-03-02-10-15-31}

