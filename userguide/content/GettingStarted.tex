% This file is part of INVERITA.
%
% INVERITA Personal Backup Software
% Copyright (C) 2012-2013  Marc Weidler <marc.weidler@web.de>,
%                          Ulrichstr. 12/1, 71672 Marbach, Germany.
%                          All rights reserved.
%
% INVERITA is free software: you can redistribute it and/or modify
% it under the terms of the GNU General Public License as published by
% the Free Software Foundation, either version 3 of the License, or
% (at your option) any later version.
%
% INVERITA is distributed in the hope that it will be useful,
% but WITHOUT ANY WARRANTY; without even the implied warranty of
% MERCHANTABILITY or FITNESS FOR A PARTICULAR PURPOSE.  See the
% GNU General Public License for more details.
%
% You should have received a copy of the GNU General Public License
% along with this program.  If not, see <http://www.gnu.org/licenses/>.


\chapter{Getting started}

\section{Basic functionality}

INVERITA is a Software


\section{Terminology}

This section explains important terms which is needed to understand
the functionality of Inverita.

\subsection{Backup}
   Inverita can manage several Backups. Each Backup has
         a storage location where the configuration and the
         Backup snapshots are stored.
         Inverita stores the list of Backup in the user's
         configuration directory \monospace{\$HOME/.config/inverita.conf}.

\subsection{Configuration}
         The Backup Configuration specifies the name of the Backup,
         if the data should be encrypted or not, included
         directories, exclude patterns and many more options.
         The configuration file is stored in the backup storage
         location and has the name 'inverita.conf'.

\subsection{Backup snapshot}
         The Backup snapshot is a directory inside the
         backup storage and contains meta data (number of files, sizes, etc.),
         cryptographic digests of all backed up files and the backed up files itself.
         Identical files are linked in all backup snapshots and thus do not
         require additional storage space.
         The directory names of backup snapshots start with a @-sign followed by
         the time stamp when the backup snapshot has been created, e.g. \monospace{@2013-03-02-10-15-31}

